\documentclass[12pt]{amsart}

\input{preamble}
\renewcommand{\thesubsection}{\arabic{subsection}}

\DeclareMathOperator{\Hom}{Hom}
\DeclareMathOperator{\Gal}{Gal}
\newcommand{\ot}{\otimes}
\newcommand{\op}{\oplus}
\newcommand{\id}{\mathrm{id}}

\begin{document}

\title{Short notes on Generalized Kummer Theory}
\maketitle

\subsection{Reference:} \cite[\S 4.10]{bos18}.

\subsection{Data:}
\begin{enumerate}
    \item Let $K$ be a field and fix a separable closure $K_{s}$.
    \item Let $G:=\Gal(K_{s}/K)$ be the absolute Galois group.
    \item Let $A$ be an abelian group endowed with the discrete topology and a continuous action of $G$ on $A$ via group automorphisms, which we will denote by $\sigma \cdot a=:\sigma(a)$.
    \item Continuity ensures that for all $a\in A$ we have
	\begin{center}
	    \begin{tikzcd}
		G(A/a):=\{ \sigma \in G\mid \sigma(a)=a\}\arrow[open]{r} & G.
	    \end{tikzcd}
	\end{center}
\end{enumerate}

It follows from $(4)$ that $G(A/a)$ is also closed in $G$, hence corresponds to an intermediate field $K\subseteq K_{s}^{G(A/a)}\subseteq K_{s}$ \cite[4.2/3]{bos18}, let's denote it $K(a)$.

\begin{lm}
    The intermediate field $K(a)$ is a finite extension of $K$.
    \begin{proof}
	Let $\{ L_{i} \}_{i\in I}$ be the direct system of all subfields of $K_{s}$ which are finite field extensions of $K$.
	For each $i\in I$, let us denote by
	\[ f_{i}\colon G\to \Gal(L_{i}/K) \]
	the restriction morphism.
	The topology in $G$ is the coarsest one making all the $f_{i}$ continuous.
	Since each $\Gal(L_{i}/K)$ is a finite group, endowed with the discrete topology, it follows that the topology on $G$ should be the smallest topology in which all fibres of the morphisms $f_{i}$ are open.
	But the fibres of all the $f_{i}$ already form a basis for some topology on $G$, so the topology on $G$ can be explicitly described in terms of this basis.

	Since $G(A/a)$ is open and $\id_{K_{s}}\in G(A/a)$, there is some $i\in I$ such that
	\[ f_{i}^{-1}(f_{i}(\id_{K_{s}}))=\Gal(K_{s}/L_{i})\subseteq G(A/a). \]
	From Galois correspondence we deduce now that
	\[ K\subseteq K(a)\subseteq L_{i}, \]
	hence $K(a)$ is also finite over $K$.
    \end{proof}
\end{lm}

\newpage
\bibliographystyle{alpha}
\bibliography{main}
\vfill

\end{document}

