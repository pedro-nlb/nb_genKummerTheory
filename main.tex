\documentclass[12pt]{amsart}


\usepackage{libertine}
\usepackage[libertine]{newtxmath}
% Local font definition; before fontenc, cf. https://tex.stackexchange.com/a/2867

\usepackage[T1]{fontenc}
% This uses 8-bit font encoding (with 256 glyphs) instead of the default 7-bit font encoding (with 128 glyphs). For example, with this option ö is a single glyph in the font, whereas on the 7-bit font encoding the font ö is made by adding an accent to the existing glyph o. A bad consequence of not using this package is that you cannot properly copy-paste such words form the output pdf file. Also, for some reason, funny stuff happens with |, < and > in text.
% Some people suggest to load fontenc before inputenc, most agree that it does not matter.

\usepackage[utf8]{inputenc}
% When you type ä in an editor set up for utf8, the machine stores the character number 228. When TeX reads the file it finds the character number 228 and the macros of inputenc transform this into \"a. Finally fontenc does its thing and transforms this into the command print character 228 (otherwise the two things would be printed separatedly as explained in fontenc).

\usepackage{mathtools}
% Loads the amsmath package (\usepackage{amsmath}: miscellaneous improvements such as the commands \DeclareMathOperator and \text). It fixes some quirks it has and adds some useful settings, symbols and environments. It improves the aesthetics as well.

%\usepackage{amssymb}
% Extended symbol collection, e.g. \Cap and \Cup. More importantly: the \mathbb command! It loads the amsfonts package (\usepackage{amsfonts}: fraktur letters, bold Greek letters...), so we do not need to include it in the preamble anymore.

\usepackage{mathrsfs}
% Font package (only supports upper case letters).

\usepackage{enumitem}
% To control the layout of enumerate, itemize and description. It supersedes the enumerate package.

\usepackage{tikz-cd}
% To draw commutative diagrams.
\usetikzlibrary{decorations.markings}
% For open and closed immersions.

\usepackage{graphicx}
% An extension of the graphics package, with optional arguments for the \includegraphics command.

\usepackage{todonotes}
% To write to do notes use the command \todo.

\usepackage{xcolor}
% To write in colors.

\usepackage{marginnote}
% To write on margins.

\usepackage{manfnt}
% To draw dangerous bent symbol.

\usepackage{float}
% Improved interface for floating objects such as figures and tables, introducing for example the H modifier to force the position of a float in the page or the boxed float. Should be loaded before hyperref.

\usepackage{hyperref}
% To handle cross-referencing and produce hypertext links in the document. It should be loaded last (with few exceptions), because it redefines many LaTeX commands.
% The backref option would insert links on each bibliography entry to the pages in which the citation was used.
%% The hidelinks option would remove colors and boxes around links, but the links remain clickable. On firefox the links are even highlighted when the mouse pointer passes over them.
%\renewcommand{\backref}[1]{$\uparrow$~#1}
% Woudl add an upwards arrow before referencing to the pages in which the citations appear.

\usepackage[noabbrev]{cleveref}
% Enhances cross-referencing features, e.g. to reference to a theorem and automatically include the word theorem.
% No abbreviature option to write figure instead of fig. etc.

%%% General things

% Custom colors
\definecolor{darkgreen}{RGB}{0,75,0}
\definecolor{darkblue}{RGB}{0,0,75}
\definecolor{darkred}{RGB}{75,0,0}
\definecolor{linkred}{rgb}{0.6,0.2,0.2}
\definecolor{linkblue}{rgb}{0,0.2,0.6}
\definecolor{linkgreen}{rgb}{0.2,0.6,0.2}

% Limit table of contents to section titles
\setcounter{tocdepth}{1}

% Sloppy formatting -- often looks better
\sloppy

%%% Font definitions

% Script Font used for sheaves
\DeclareFontFamily{OMS}{rsfs}{\skewchar\font'60}
\DeclareFontShape{OMS}{rsfs}{m}{n}{<-5>rsfs5 <5-7>rsfs7 <7->rsfs10 }{}
\DeclareSymbolFont{rsfs}{OMS}{rsfs}{m}{n}
\DeclareSymbolFontAlphabet{\scr}{rsfs}
\DeclareSymbolFontAlphabet{\scr}{rsfs}

% Sheaves
\newcommand{\sA}{\scr{A}}
\newcommand{\sB}{\scr{B}}
\newcommand{\sC}{\scr{C}}
\newcommand{\sD}{\scr{D}}
\newcommand{\E}{\scr{E}} % Exception (Vector bundles)
\newcommand{\F}{\scr{F}} % Exception (Coherent sheaves)
\newcommand{\G}{\scr{G}} % Exception (Coherent sheaves)
\newcommand{\sH}{\scr{H}}
\renewcommand{\hom}{\scr{H}\negthinspace om} % Exception (Hom-sheaf)
\newcommand{\I}{\scr{I}} % Exception (Ideal sheaves)
\newcommand{\sJ}{\scr{J}}
\newcommand{\sK}{\scr{K}}
\renewcommand{\L}{\scr{L}} % Exception (Line bundles)
\newcommand{\M}{\scr{M}} % Exception (Line bundles)
\newcommand{\sN}{\scr{N}}
\renewcommand{\O}{\scr{O}} % Exception (Structure sheaf)
\newcommand{\sP}{\scr{P}}
\newcommand{\sQ}{\scr{Q}}
\newcommand{\sR}{\scr{R}}
\newcommand{\sS}{\scr{S}}
\newcommand{\sT}{\scr{T}}
\newcommand{\sU}{\scr{U}}
\newcommand{\sV}{\scr{V}}
\newcommand{\sW}{\scr{W}}
\newcommand{\w}{\omega} % Addition (Canonical sheaf)
\newcommand{\sX}{\scr{X}}
\newcommand{\sY}{\scr{Y}}
\newcommand{\sZ}{\scr{Z}}

% Mathcal fonts
\newcommand{\calA}{\mathcal{A}}
\newcommand{\calB}{\mathcal{B}}
\newcommand{\calC}{\mathcal{C}}
\newcommand{\calD}{\mathcal{D}}
\newcommand{\calE}{\mathcal{E}}
\newcommand{\calF}{\mathcal{F}}
\newcommand{\calG}{\mathcal{G}}
\newcommand{\calH}{\mathcal{H}}
\newcommand{\calI}{\mathcal{I}}
\newcommand{\calJ}{\mathcal{J}}
\newcommand{\calK}{\mathcal{K}}
\newcommand{\calL}{\mathcal{L}}
\newcommand{\calM}{\mathcal{M}}
\newcommand{\calN}{\mathcal{N}}
\newcommand{\calO}{\mathcal{O}}
\newcommand{\calP}{\mathcal{P}}
\newcommand{\calQ}{\mathcal{Q}}
\newcommand{\calR}{\mathcal{R}}
\newcommand{\calS}{\mathcal{S}}
\newcommand{\calT}{\mathcal{T}}
\newcommand{\U}{\mathcal{U}} % Exception (Open covers)
\newcommand{\calV}{\mathcal{V}}
\newcommand{\calW}{\mathcal{W}}
\newcommand{\X}{\mathcal{X}} % Exception (Families of varieties)
\newcommand{\Y}{\mathcal{Y}} % Exception (Families of varieties)
\newcommand{\calZ}{\mathcal{Z}}

% Blackboard Bold Symbols
\newcommand{\A}{\mathbb{A}} % Exception (Affine space)
\newcommand{\bbB}{\mathbb{B}}
\newcommand{\C}{\mathbb{C}} % Exception (Complex numbers)
\newcommand{\bbD}{\mathbb{D}}
\newcommand{\bbE}{\mathbb{E}}
\newcommand{\bbF}{\mathbb{F}}
\newcommand{\bbG}{\mathbb{G}}
\newcommand{\Gm}{\mathbb{G}_{\mathrm{m}}} % Addition (Punctured affine line)
\newcommand{\bbH}{\mathbb{H}}
\newcommand{\bbI}{\mathbb{I}}
\newcommand{\bbJ}{\mathbb{J}}
\newcommand{\bbK}{\mathbb{K}}
\newcommand{\bbL}{\mathbb{L}}
\newcommand{\bbM}{\mathbb{M}}
\newcommand{\N}{\mathbb{N}} % Exception (Natural numbers)
\newcommand{\bbO}{\mathbb{O}}
\renewcommand{\P}{\mathbb{P}} % Exception (Projective space)
\newcommand{\Q}{\mathbb{Q}} % Exception (Rational numbers)
\newcommand{\R}{\mathbb{R}} % Exception (Real numbers)
\newcommand{\bbS}{\mathbb{S}}
\newcommand{\bbT}{\mathbb{T}}
\newcommand{\bbU}{\mathbb{U}}
\newcommand{\V}{\mathbb{V}} % Exception (Geometric vector bundle)
\newcommand{\bbW}{\mathbb{W}}
\newcommand{\bbX}{\mathbb{X}}
\newcommand{\bbY}{\mathbb{Y}}
\newcommand{\Z}{\mathbb{Z}} % Exception (Integers)

% Boldfont (categories)
\newcommand{\bfA}{\mathbf{A}}
\newcommand{\Ab}{\mathbf{Ab}}
\newcommand{\bfB}{\mathbf{B}}
\newcommand{\bfC}{\mathbf{C}}
\newcommand{\Cat}{\mathbf{Cat}} % Addition (Categories)
\newcommand{\Coh}{\mathbf{Coh}} % Addition (Coherent sheaves)
\newcommand{\D}{\mathbf{D}} % Exception (Derived category)
\newcommand{\Db}{\mathbf{D}^{\mathrm{b}}} % Addition (Bounded derived category)
\newcommand{\bfE}{\mathbf{E}}
\newcommand{\bfF}{\mathbf{F}}
\newcommand{\bfG}{\mathbf{G}}
\newcommand{\bfH}{\mathbf{H}}
\newcommand{\bfI}{\mathbf{I}}
\newcommand{\bfJ}{\mathbf{J}}
\newcommand{\K}{\mathbf{K}} % Exception (Homotopy category)
\newcommand{\bfL}{\mathbf{L}}
\newcommand{\bfM}{\mathbf{M}}
\newcommand{\Mod}{\mathbf{Mod}} % Addition (Modules)
\newcommand{\bfN}{\mathbf{N}}
\newcommand{\bfO}{\mathbf{O}}
\newcommand{\bfP}{\mathbf{P}}
\newcommand{\PSh}{\mathbf{PSh}} % Addition (Presheaves)
\newcommand{\bfQ}{\mathbf{Q}}
\newcommand{\QCoh}{\mathbf{QCoh}} % Addition (Quasi-coherent sheaves)
\newcommand{\bfR}{\mathbf{R}}
\newcommand{\bfS}{\mathbf{S}}
\newcommand{\Set}{\mathbf{Set}} % Addition (Sets)
\newcommand{\Sh}{\mathbf{Sh}} % Addition (Sheaves)
\newcommand{\bfT}{\mathbf{T}}
\newcommand{\bfU}{\mathbf{U}}
\newcommand{\bfV}{\mathbf{V}}
\renewcommand{\Vec}{\mathbf{Vec}} % Addition (Vector bundles)
\newcommand{\bfW}{\mathbf{W}}
\newcommand{\bfX}{\mathbf{X}}
\newcommand{\bfY}{\mathbf{Y}}
\newcommand{\bfZ}{\mathbf{Z}}

% Mathfrak for ideals
\renewcommand{\a}{\mathfrak{a}}
\renewcommand{\b}{\mathfrak{b}}
\renewcommand{\c}{\mathfrak{c}}
\renewcommand{\d}{\mathfrak{d}}
\newcommand{\e}{\mathfrak{e}}
\newcommand{\m}{\mathfrak{m}}
\newcommand{\n}{\mathfrak{n}}

%%% Theorem environments

% Custom theorem styles (empty fields take default values)
\newtheoremstyle{darkgreentheorem}% name of the style
{}% measure of space to leave above the theorem. E.g.: 3pt
{}% measure of space to leave below the theorem. E.g.: 3pt
{\itshape}% name of font to use in the body of the theorem
{}% measure of space to indent
{\color{darkgreen}\bfseries}% name of head font
{.}% punctuation between head and body
{ }% space after theorem head; " " = normal interword space
{}% Manually specify head
\newtheoremstyle{darkbluedefinition}
{}{}{}{}{\color{darkblue}\bfseries}{.}{ }{}
\newtheoremstyle{darkredexample}
{}{}{}{}{\color{darkred}\bfseries}{.}{ }{}

% Numbered theorems
\theoremstyle{plain}
% \theoremstyle{darkgreentheorem}
\newtheorem{thm}{Theorem}
\newtheorem{lm}[thm]{Lemma}
\newtheorem{prop}[thm]{Proposition}
\newtheorem{cor}[thm]{Corollary}
\newtheorem{conj}[thm]{Conjecture}
\newtheorem{fact}[thm]{Fact}
\newtheorem{axiom}[thm]{Axiom}
\theoremstyle{definition}
% \theoremstyle{darkbluedefinition}
\newtheorem{defn}[thm]{Definition}
% \theoremstyle{darkredexample}
\newtheorem{exa}[thm]{Example}
\theoremstyle{remark}
\newtheorem{rem}[thm]{Remark}
\newtheorem{nota}[thm]{Notation}
\newtheorem{q}[thm]{Question}
\newtheorem{exe}[thm]{Exercise}

% Custom numbered theorems
\theoremstyle{plain}
% \theoremstyle{darkgreentheorem}
\newtheorem{innercustomthm}{Theorem}
\newenvironment{cthm}[1]
    {\renewcommand\theinnercustomthm{#1}\innercustomthm}
    {\endinnercustomthm}
\newtheorem{innercustomlm}{Lemma}
\newenvironment{clm}[1]
    {\renewcommand\theinnercustomlm{#1}\innercustomlm}
    {\endinnercustomlm}
\newtheorem{innercustomprop}{Proposition}
\newenvironment{cprop}[1]
    {\renewcommand\theinnercustomprop{#1}\innercustomprop}
    {\endinnercustomprop}
\newtheorem{innercustomcor}{Corollary}
\newenvironment{ccor}[1]
    {\renewcommand\theinnercustomcor{#1}\innercustomcor}
    {\endinnercustomcor}
\newtheorem{innercustomconj}{Conjecture}
\newenvironment{cconj}[1]
    {\renewcommand\theinnercustomconj{#1}\innercustomconj}
    {\endinnercustomconj}
\newtheorem{innercustomfact}{Fact}
\newenvironment{cfact}[1]
    {\renewcommand\theinnercustomfact{#1}\innercustomfact}
    {\endinnercustomfact}
% Definitions
\theoremstyle{definition}
% \theoremstyle{darkbluedefinition}
\newtheorem{innercustomdefn}{Definition}
\newenvironment{cdefn}[1]
    {\renewcommand\theinnercustomdefn{#1}\innercustomdefn}
    {\endinnercustomdefn}
% \theoremstyle{darkredexample}
\newtheorem{innercustomexa}{Example}
\newenvironment{cexa}[1]
    {\renewcommand\theinnercustomexa{#1}\innercustomexa}
    {\endinnercustomexa}
\theoremstyle{remark}
\newtheorem{innercustomrem}{Remark}
\newenvironment{crem}[1]
    {\renewcommand\theinnercustomrem{#1}\innercustomrem}
    {\endinnercustomrem}
\newtheorem{innercustomnota}{Notation}
\newenvironment{cnota}[1]
    {\renewcommand\theinnercustomnota{#1}\innercustomnota}
    {\endinnercustomnota}
\newtheorem{innercustomq}{Question}
\newenvironment{cq}[1]
    {\renewcommand\theinnercustomq{#1}\innercustomq}
    {\endinnercustomq}
\newtheorem{innercustomexe}{Exercise}
\newenvironment{cexe}[1]
    {\renewcommand\theinnercustomexe{#1}\innercustomexe}
    {\endinnercustomexe}

% Unnumbered theorems
\theoremstyle{plain}
% \theoremstyle{darkgreentheorem}
\newtheorem*{uthm}{Theorem}
\newtheorem*{ulm}{Lemma}
\newtheorem*{uprop}{Proposition}
\newtheorem*{ucor}{Corollary}
\newtheorem*{uconj}{Conjecture}
\newtheorem*{ufact}{Fact}
\theoremstyle{definition}
% \theoremstyle{darkbluedefinition}
\newtheorem*{udefn}{Definition}
% \theoremstyle{darkredexample}
\newtheorem*{uexa}{Example}
\theoremstyle{remark}
\newtheorem*{urem}{Remark}
\newtheorem*{unota}{Notation}
\newtheorem*{uq}{Question}
\newtheorem*{uexe}{Exercise}

% Cross-referencing
\crefname{thm}{theorem}{theorems}
\Crefname{thm}{Theorem}{Theorems}
\crefname{lm}{lemma}{lemmas}
\Crefname{lm}{Lemma}{Lemmas}
\crefname{prop}{proposition}{propositions}
\Crefname{prop}{Proposition}{Propositions}
\crefname{cor}{corollary}{corollaries}
\Crefname{cor}{Corollary}{Corollaries}
\crefname{conj}{conjecture}{conjectures}
\Crefname{conj}{Conjecture}{Conjectures}
\crefname{fact}{fact}{facts}
\Crefname{fact}{Fact}{Facts}
\crefname{defn}{definition}{definitions}
\Crefname{defn}{Definition}{Definitions}
\crefname{exa}{example}{examples}
\Crefname{exa}{Example}{Examples}
\crefname{rem}{remark}{remarks}
\Crefname{rem}{Remark}{Remarks}
\crefname{nota}{notation}{notations}
\Crefname{nota}{Notation}{Notations}
\crefname{q}{question}{questions}
\Crefname{q}{Question}{Questions}
\crefname{exe}{exercise}{exercises}
\Crefname{exe}{Exercise}{Exercises}
% More cross-referencing
\crefname{cthm}{theorem}{theorems}
\Crefname{cthm}{Theorem}{Theorems}
\crefname{clm}{lemma}{lemmas}
\Crefname{clm}{Lemma}{Lemmas}
\crefname{cprop}{proposition}{propositions}
\Crefname{cprop}{Proposition}{Propositions}
\crefname{ccor}{corollary}{corollaries}
\Crefname{ccor}{Corollary}{Corollaries}
\crefname{cconj}{conjecture}{conjectures}
\Crefname{cconj}{Conjecture}{Conjectures}
\crefname{cfact}{fact}{facts}
\Crefname{cfact}{Fact}{Facts}
\crefname{cdefn}{definition}{definitions}
\Crefname{cdefn}{Definition}{Definitions}
\crefname{cexa}{example}{examples}
\Crefname{cexa}{Example}{Examples}
\crefname{crem}{remark}{remarks}
\Crefname{crem}{Remark}{Remarks}
\crefname{cnota}{notation}{notations}
\Crefname{cnota}{Notation}{Notations}
\crefname{cq}{question}{questions}
\Crefname{cq}{Question}{Questions}
\crefname{cexe}{exercise}{exercises}
\Crefname{cexe}{Exercise}{Exercises}

%%% Tikzcd

% Open and closed immersion arrows.
\makeatletter
\tikzcdset{
open/.code={\tikzcdset{hook, circled};},
closed/.code={\tikzcdset{hook, slashed};},
circled/.code={\tikzcdset{markwith={\draw (0,0) circle (.375ex);}};},
slashed/.code={\tikzcdset{markwith={\draw[-] (-.4ex,-.4ex) -- (.4ex,.4ex);}};},
markwith/.code={
\pgfutil@ifundefined{tikz@library@decorations.markings@loaded}%
{\pgfutil@packageerror{tikz-cd}{You need to say %
\string\usetikzlibrary{decorations.markings} to use arrow with markings}{}}{}%
\pgfkeysalso{/tikz/postaction={/tikz/decorate,
/tikz/decoration={
markings,
mark = at position 0.5 with
{#1}}}}},
}
\makeatother

%%% Author, title, etc.

% Author info
\author[Pedro N\'{u}\~{n}ez]{}
\address{Pedro N\'{u}\~{n}ez \newline
\indent Albert-Ludwigs-Universit\"{a}t Freiburg, Mathematisches Institut \newline
\indent Ernst-Zermelo-Straße 1, 79104 Freiburg im Breisgau (Germany)}
\email{\href{mailto:pedro.nunez@math.uni-freiburg.de}{pedro.nunez@math.uni-freiburg.de}}
\renewcommand*{\urladdrname}{\itshape Homepage}
\urladdr{\href{https://home.mathematik.uni-freiburg.de/nunez/}{https://home.mathematik.uni-freiburg.de/nunez} \newline
\mbox{ }\newline 
\indent {\fontfamily{libertine}\selectfont
I would like to thank the DFG-Graduiertenkolleg GK1821 ``Cohomological Methods in Geometry'' for their support!%
}}
%\thanks{The author gratefully acknowledges support by the DFG-Graduiertenkolleg GK1821 ``Cohomological Methods in Geometry'' at the University of Freiburg.}

% Content details
%\keywords{...}
%\subjclass[...]{...}
\title[Notes]{Notes}
\date{\today}

% Links and pdf options
\makeatletter
\hypersetup{
  pdfauthor={\authors},
  pdftitle={\@title},
  %pdfsubject={\@subjclass},
  %pdfkeywords={\@keywords},
  pdfstartview={Fit},
  pdfpagelayout={TwoColumnRight},
  pdfpagemode={UseOutlines},
  bookmarks,
  colorlinks,
  linkcolor=linkblue,
  citecolor=linkblue,
  urlcolor=linkblue}
\makeatother



\DeclareMathOperator{\Hom}{Hom}
\DeclareMathOperator{\Gal}{Gal}
\DeclareMathOperator{\Aut}{Aut}
\newcommand{\ot}{\otimes}
\newcommand{\op}{\oplus}
\newcommand{\id}{\mathrm{id}}

\begin{document}

\title{Short notes on General Kummer Theory}
\maketitle

\section{Preliminaries}

\subsection*{Goal:} given a field $K$ and a non-zero natural number $n$, characterize all Galois extensions of $K$ whose Galois group is abelian with exponent $d\mid n$.

\subsection*{Language:} by \textit{abelian} extension we mean a Galois extension $L/K$ with abelian Galois group; by \textit{cyclic} extension we mean a Galois extension $L/K$ with cyclic Galois group.

\subsection*{Reference:} \cite[\S 4.10]{bos18}.

\section{Setting}

\begin{enumerate}
    \item Let $K$ be a field and fix a separable closure $K_{s}$.
    \item Let $n\in \N$ be a non-zero natural number.
    \item Let $G:=\Gal(K_{s}/K)$ be the absolute Galois group.
    \item Let $A$ be an abelian group endowed with the discrete topology and a continuous action of $G$ on $A$ via group automorphisms, which we will denote by $\sigma \cdot a=:\sigma(a)$.
    \item For each intermediate field $K\subseteq L\subseteq K_{s}$ we denote
	\[ A_{L}:=\{ a\in A\mid \sigma(a)=a \text{ for all }\sigma\in \Gal(K_{s}/L)\}. \]
    \item Let $\wp\colon A\to A$ be a $G$-equivariant surjective homomorphism whose kernel, denoted $\mu_{n}$, is a cyclic subgroup of order $n$ of $A_{K}$.
\end{enumerate}

Continuity of the action of $G$ on $A$ ensures that for all $a\in A$ we have
\begin{center}
    \begin{tikzcd}
	G(A/a):=\{ \sigma \in G\mid \sigma(a)=a\}\arrow[open]{r} & G.
    \end{tikzcd}
\end{center}
Hence $G(A/a)$ is also closed in $G$ and corresponds to an intermediate field $K\subseteq K_{s}^{G(A/a)}\subseteq K_{s}$ \cite[4.2/3]{bos18}, let's denote it $K(a)$.

\begin{lm}\label{lm:intermediate}
    The intermediate field $K(a)$ is a finite extension of $K$.
    \begin{proof}
	Let $\{ L_{i} \}_{i\in I}$ be the direct system of all subfields of $K_{s}$ which are finite field extensions of $K$.
	For each $i\in I$, let us denote by
	\[ f_{i}\colon G\to \Gal(L_{i}/K) \]
	the restriction morphism.
	The topology in $G$ is the coarsest one making all the $f_{i}$ continuous.
	Since each $\Gal(L_{i}/K)$ is a finite group, endowed with the discrete topology, it follows that the topology on $G$ should be the smallest topology in which all fibres of the morphisms $f_{i}$ are open.
	But the fibres of all the $f_{i}$ already form a basis for some topology on $G$, so the topology on $G$ can be explicitly described in terms of this basis.

	Since $G(A/a)$ is open and $\id_{K_{s}}\in G(A/a)$, there is some $i\in I$ such that
	\[ f_{i}^{-1}(f_{i}(\id_{K_{s}}))=\Gal(K_{s}/L_{i})\subseteq G(A/a). \]
	From Galois correspondence we deduce now that
	\[ K\subseteq K(a)\subseteq L_{i}, \]
	hence $K(a)$ is also finite over $K$.
    \end{proof}
\end{lm}

More generally, given a subset $\Delta\subseteq A$ we may consider the subgroup
\[ G(A/\Delta):=\{\sigma\in G\mid \sigma(a)=a \text{ for all }a\in \Delta\}=\cap_{a\in \Delta}G(A/a), \]
which is then a closed subgroup but not necessarily an open subgroup.
In any case we obtain an intermediate field $K\subseteq K_{s}^{G(A/\Delta)}\subseteq K_{s}$, which we will denote by $K(\Delta)$.

If $L/K$ is Galois, then the action of $G$ on $A$ restricts to an action of $G$ on $A_{L}$.
Indeed, let $\tau\in G$, $\sigma\in \Gal(K_{s}/L)$ and $a\in A_{L}$.
Since $\Gal(K_{s}/L)\trianglelefteq G$, there is some $\sigma'\in \Gal(K_{s}/L)$ such that
\[ \sigma\tau(a)=\tau\sigma'(a)=\tau(a), \]
hence $\tau(a)\in A_{L}$.
And by definition $\Gal(K_{s}/L)$ acts trivially on $A_{L}$, so we get an induced action of $G/\Gal(K_{s}/L)$ on $A_{L}$.
Using again that $L/K$ is Galois, we may identify this quotient group with $\Gal(L/K)$, obtaining an action of $\Gal(L/K)$ on $A_{L}$.
We can then talk about the cohomology group $H^{1}(\Gal(L/K),A_{L})$.
A function $f\colon \Gal(L/K)\to A_{L}$ is called a \textit{crossed homomorphism} if for all $\sigma,\tau\in \Gal(L/K)$ we have
\[ f(\sigma\tau)=f(\sigma)+\sigma (f(\tau)). \]
A function $f\colon \Gal(L/K)\to A_{L}$ is called a \textit{principal crossed homomorphism} if there exists some $a\in A_{L}$ such that for all $\sigma \in \Gal(L/K)$ we have
\[ f(\sigma)=\sigma(a)-a. \]
Principal crossed homomorphisms form a subgroup of the group of crossed homomorphisms, and the quotient group is then our first cohomology group $H^{1}(\Gal(L/K),A_{L})$.

We are ready now to state the main assumption on which we will rely:

\begin{axiom}
    For every cyclic extension $L/K$ whose degree divides $n$ we have
    \[ H^{1}(\Gal(L/K),A_{L})=0. \]
\end{axiom}

\section{The pairing associated to a subgroup}

Let $C\subseteq A_{K}$ be a subgroup and consider $\wp^{-1}(C)\subseteq A$.
By $G$-equivariance of $\wp$ and our assumption that $C\subseteq A_{K}$, any $\sigma\in G$ restricts to a homomorphism $\sigma\colon \wp^{-1}(C)\to \wp^{-1}(C)$.
If $\sigma(a)=0$ for $a\in \wp^{-1}(C)$, then
\[ \wp(\sigma(a))=\sigma(\wp(a))=\wp(a)=0, \]
because $\wp(a)\in C\subseteq A_{K}$.
Therefore $a\in \mu_{n}\subseteq A_{K}$, and this implies in turn that $\sigma(a)=a=0$.
So the restriction of $\sigma$ is an injective homomorphism $\wp^{-1}(C)\to \wp^{-1}(C)$.
For $a\in \wp^{-1}(C)$ we have
\[ \sigma(a)-a\in \mu_{n} \]
again by $G$-equivariance of $\wp$ and our assumption that $C\subseteq A_{K}$.
So if $\sigma(a)\in \wp^{-1}(C)$, then
\[ \wp(\sigma(a))=\wp(a)\in C \]
and $a\in \wp^{-1}(C)$ as well, showing that the restriction of $\sigma$ is also surjective.
Hence $\sigma$ restricts to a bijection $\wp^{-1}(C)\to \wp^{-1}(C)$.
We obtain in this manner a group homomorphism
\[ G\to \Aut(\wp^{-1}(C)). \]
The kernel of this group homomorphism is $G(A/\wp^{-1}(C))$ by definition.
It is therefore a normal subgroup of $G$, which means in turn that $K(\wp^{-1}(C))/K$ is a Galois extension with Galois group $G_{C}\cong G/G(A/\wp^{-1}(C))$.
In particular, we also obtain an induced action of the Galois group $G_{C}$ on $\wp^{-1}(C)$.

We define now a pairing
\begin{align*}
    G_{C}\times C &\longrightarrow \mu_{n} \\
    (\sigma,c) &\longmapsto \sigma(a)-a, \text{ for }a\in \wp^{-1}(c).
\end{align*}
To check that it is well-defined, pick some other $a'\in \wp^{-1}(c)$.
This elemnt will differ from the previous $a$ by some $b\in \mu_{n}$, hence
\[ \sigma(a')-a'=\sigma(a)+\sigma(b)-a-b=\sigma(a)-a. \]
All good then.
Assume from now on that $\wp(A_{K})\subseteq C$.
We factor then the previous pairing into the pairing that we are interested in:
\begin{align*}
    \langle \cdot,\cdot\rangle \colon G_{C}\times C/\wp(A_{K}) &\longrightarrow \mu_{n} \\
    (\sigma,\bar{c}) & \longmapsto \sigma(a)-a, \text{ for }a\in \wp^{-1}(c).
\end{align*}

\begin{prop}\label{prop:injectivity}
    The pairing $\langle \cdot,\cdot\rangle$ is non-degenerate.
    \begin{proof}
	We have to show that the induced morphisms
	\[ \varphi_{1}\colon G_{C}\to \Hom(C/\wp(A_{K}),\mu_{n}) \quad\text{and}\quad \varphi_{2}\colon C/\wp(A_{K})\to \Hom(G_{C},\mu_{n}) \]
	are injective.

	Suppose that $\sigma\in G_{C}$ is such that $\langle \sigma,\bar{c}\rangle=0$ for all $\bar{c}\in C/\wp(A_{K})$.
	In particular, if $\sigma'\in G$ is a preimage of $\sigma$, then $\sigma(a)=a$ for all $a\in \wp^{-1}(C)$.
	This means precisely that $\sigma'\in G(A/\wp^{-1}(C))$, hence $\sigma=1_{G_{C}}$.

	Suppose now that $c\in C$ is such that $\langle \sigma, \bar{c}\rangle=0$ for all $\sigma\in G_{C}$.
	We want to show that $c\in \wp(A_{K})$, so let $a\in \wp^{-1}(c)$.
	For all $\sigma'\in G$ we have $\sigma'(a)=a$, which means that $a\in A_{K}$ and therefore $\bar{c}=0$.
    \end{proof}
\end{prop}

\begin{prop}\label{prop:finiteness}
    $K(\wp^{-1}(C))/K$ is finite if and only if $(C:\wp(A_{K}))$ is finite.
    \begin{proof}
	Suppose first that $[K(\wp^{-1}(C)):K]$ is finite.
	Then its Galois group $G_{C}$ would be finite as well, so $\Hom(G_{C},\mu_{n})$ is finite.
	But $\varphi_{2}$ is injective by \Cref{prop:injectivity}, so $C/\wp(A_{K})$ must be finite as well.

	Conversely, suppose that $C/\wp(A_{K})$ is finite.
	Again, this implies that $\Hom(C/\wp(A_{K}),\mu_{n})$ is finite, so injectivity of $\varphi_{1}$ shows that $[K(\wp^{-1}(C)):K]$ is finite as well.
    \end{proof}
\end{prop}

\begin{lm}\label{lm:finiteduality}
    Let $n\in \N$ be a non-zero natural number and let $H$ be a finite abelian group with exponent $d\mid n$.
    Then there exists an isomorphism $H\cong \Hom(H,\Z/n\Z)$.
    \begin{proof}
	By the structure theorem for finitely generated abelian groups it suffices to show the result for $H=\Z/d\Z$.
	We first reduce the result to the case $d=n$.
	There is a unique cyclic subgroup $H_{d}\subseteq \Z/n\Z$ of order $d$.
	Every homomorphism $\Z/d\Z\to \Z/n\Z$ factors then through $H_{d}$, so the canonical map
	\[ \Hom(\Z/d\Z,H_{d})\hookrightarrow \Hom(\Z/d\Z,\Z/n\Z) \]
	is an isomorphism.
	Since $H_{d}\cong \Z/d\Z$, it suffices to show that there is an isomorphism
	\[ \Z/d\Z\to \Hom(\Z/d\Z,\Z/d\Z), \]
	i.e.~it suffices to show the case $d=n$.

	In this case we consider the surjective homomorphism
	\begin{align*}
	    \Z & \to \Hom(\Z/d\Z,\Z/d\Z) \\
	    1 & \mapsto \id.
	\end{align*}
	Its kernel is $d\Z$, so passing to the quotient yields the desired isomorphism.
    \end{proof}
\end{lm}

\begin{prop}\label{prop:finitecase}
    If $K(\wp^{-1}(C))/K$ or $(C:\wp(A_{K}))$ are finite, then $\varphi_{1}$ and $\varphi_{2}$ from \Cref{prop:injectivity} are isomorphisms and
    \[ [K(\wp^{-1}(C)):K]=(C:\wp(A_{K})). \]
    \begin{proof}
	By \Cref{prop:finiteness}, if either of the two is finite, so is the other one.
	By \Cref{lm:finiteduality} we have isomorphisms
	\[ C/\wp(A_{K})\cong \Hom(C/\wp(A_{K}),\mu_{n})\quad\text{and}\quad G_{C}\cong \Hom(G_{C},\mu_{n}). \]
	We have
	\begin{align*}
	    [K(\wp^{-1}(C)):K] & = |G_{C}| \\
	    & \leqslant |\Hom(C/\wp(A_{K}),\mu_{n})| \\
	    & = |C/\wp(A_{K})| \\
	    & \leqslant |\Hom(G_{C},\mu_{n}) | \\
	    & = |G_{C}| \\
	    & =[K(\wp^{-1}(C)):K].
	\end{align*}
	Therefore $[K(\wp^{-1}(C)):K]=(C:\wp(A_{K}))$ and $\varphi_{1}$ and $\varphi_{2}$ are isomorphisms.
    \end{proof}
\end{prop}

\begin{prop}\label{prop:infinitecase}
    Even if $[K(\wp^{-1}(C)):K]$ and $(C:\wp(A_{K}))$ are not finite, $\varphi_{1}$ is still an isomorphism and $\varphi_{2}$ induces an isomorphism
    \[ C/\wp(A_{K})\cong \Hom_{\mathrm{cont}}(G_{C},\mu_{n}) \]
    onto the subgroup of continuous homomorphisms.
    \begin{proof}
	Consider the directed system $\{ C_{i} \}_{i\in I}$ of all subgroups $C_{i}$ of $C$ containing $\wp(A_{K})$ and such that $(C_{i}:\wp(A_{K}))$ is finite.
	Since any finite intermediate extension $K\subseteq L\subseteq K(\wp^{-1}(C))$ is contained in some $K(\wp^{-1}(C_{i}))$, we can write
	\[ G_{C}\cong \varprojlim_{i\in I} G_{C_{i}}. \]
	Consider now for each $i\in I$ the commutative diagram
	\begin{center}
	    \begin{tikzcd}
		G_{C}\arrow[hook]{r}{\varphi_{1}}\arrow{d} & \Hom(C/\wp(A_{K}),\mu_{n})\arrow{d} \\
		G_{C_{i}}\arrow{r}{\cong}[swap]{\varphi_{1,i}} & \Hom(C_{i}/\wp(A_{K}),\mu_{n})
	    \end{tikzcd}
	\end{center}
	in which both horizontal arrows are the restrictions.
	Let $f\in \Hom(C/\wp(A_{K}),\mu_{n})$ and consider for each $i\in I$ its restriction $f|_{C_{i}/\wp(A_{K})}$, which comes from a unique $\sigma_{i}\in G_{C_{i}}$.
	This yields a family of automorphisms $(\sigma_{i})_{i\in I}$.
	We claim that this is a compatible family, giving therefore an element in $G_{C}$ which maps to $f$ and proving surjectivity of $\varphi_{1}$.
	To show this, let $i\leqslant j$ and consider the diagram
	\begin{center}
	    \begin{tikzcd}
		G_{C_{j}}\arrow{d}\arrow{r}{\varphi_{1,j}} & \Hom(C_{j}/\wp(A_{K}),\mu_{n})\arrow{d} \\
		G_{C_{i}}\arrow{r}{\varphi_{1,i}} & \Hom(C_{i}/\wp(A_{K}),\mu_{n})
	    \end{tikzcd}
	\end{center}
	in which both vertical arrows are the restrictions.
	We want to check that it commutes.
	The isomorphism $G/G(A/\wp^{-1}(C_{k}))\to G_{C_{k}}$ is given by restriction of automorphisms for all $k\in I$.
	So every $\tau\in G_{C_{j}}$ can be written as $\sigma|_{K(\wp^{-1}(C_{j}))}$ for some $\sigma\in G$.
	For $c\in C_{i}\subseteq C_{j}$ and $a\in \wp^{-1}(c)$ we have then
	\[ \varphi_{1,i}(\tau|_{K(\wp^{-1}(C_{i}))})(\bar{c})=\sigma(a)-a=\varphi_{1,j}(\tau)|_{C_{i}/\wp(A_{K})}(\bar{c}), \]
	showing commutativity of the diagram and thus finishing the proof of bijectivity of $\varphi_{1}$.

	Before discussing the assertion about $\varphi_{2}$, we claim that every continuous homomorphism $g\colon G_{C}\to \mu_{n}$ comes from some homomorphism $g_{i}\colon G_{C_{i}}\to \mu_{n}$ via the restriction $f_{i}\colon G_{C}\to G_{C_{i}}$.
	Indeed, let $\xi\in \mu_{n}$ be a generator.
	Given such $g$ and given some $k\xi\in \mu_{n}$ in the image of $g$, say $k\xi=g(\sigma)$, the preimage $g^{-1}(k\xi)$ is open in $G_{C}$.
	Since the fibers of the restrictions form a basis for the topology on $G_{C}$, there exists some $i_{k}\in I$ such that $f_{i_{k}}^{-1}f_{i_{k}}(\sigma)\subseteq g^{-1}(g(\sigma))$.
	Let now $i=\max_{k}\{ i_{k} \}$ and define
	\begin{align*}
	    g_{i}\colon G_{C_{i}} & \longrightarrow \mu_{n} \\
	    \sigma|_{K(\wp^{-1}(C_{i}))} & \longmapsto g(\sigma).
	\end{align*}
	If $\sigma|_{K(\wp^{-1}(C_{i}))}=\tau|_{K(\wp^{-1}(C_{i}))}$ and $g(\sigma)=k\xi$, then we have $i_{k}\leqslant i$ and therefore
	\[ \tau\in f_{i}^{-1}f_{i}(\sigma)\subseteq f_{i_{k}}^{-1}f_{i_{k}}(\sigma)\subseteq g^{-1}g(\sigma), \]
	showing that $g_{i}$ is well-defined.
	And by construction $g=g_{i}\circ f_{i}$, proving the claim.

	Moving on to the assertion about $\varphi_{2}$, suppose $g\in \Hom(G_{C},\mu_{n})$ is in the image of $\varphi_{2}$, say $g=\varphi_{2}(\bar{c})$.
	Then it is continuous, because the formula we used to define it involves only the continuous action of $G$ on $A$ and the continuous group operations in $A$.
	But we can also check this directly: by homogeneity it suffices to show that
	\[ g^{-1}(0)=\{ \sigma\in G_{C}\mid \sigma(a)-a=0 \text{ for }a\in \wp^{-1}(c) \} \]
	is closed in $G_{C}$, which is true because its preimage under the quotient map is the closed subgroup $G(A/\wp^{-1}(c))$ of $G$.
	Conversely, suppose $g\in \Hom(G_{C},\mu_{n})$ is a continuous homomorphism.
	Then by our previous claim we may find some $i\in I$ such that $g=g_{i}\circ f_{i}$ for some $g_{i}\colon G_{C_{i}}\to \mu_{n}$, where $f_{i}\colon G_{C}\to G_{C_{i}}$ denotes the restriction.
	This means that in the commutative square
	\begin{center}
	    \begin{tikzcd}
		C_{i}/\wp(A_{K}) \arrow[hook]{d}\arrow{r}{\varphi_{2,i}} & \Hom(G_{C_{i}},\mu_{n})\arrow{d} \\
		C/\wp(A_{K}) \arrow{r}{\varphi_{2}} & \Hom(G_{C},\mu_{n})
	    \end{tikzcd}
	\end{center}
	our $g$ lies in the image of the right vertical arrow.
	By \Cref{prop:finitecase} the top horizontal arrow is an isomorphism.
	Hence $g$ also lies in the image of $\varphi_{2}$.
    \end{proof}
\end{prop}

\section{Main Theorem of Kummer Theory}

Recall from the previous section that $C\subseteq A_{K}$ is a subgroup such that $\wp(A_{K})\subseteq C$ and $\{ C_{i} \}_{i\in I}$ is the directed system of all subgroups $C_{i}$ of $C$ containing $\wp(A_{K})$ such that $(C_{i}:\wp(A_{K}))$ is finite.

\begin{lm}\label{lm:fieldwelldefined}
    Then $K(\wp^{-1}(C))/K$ is an abelian extension with exponent $d$ dividing $n$.
    \begin{proof}
	This follows from injectivity of 
	\[ \Gal(K(\wp^{-1}(C))/K)\to \Hom(C/\wp(A_{K}),\mu_{n}), \]
	which was shown in \Cref{prop:injectivity}.
    \end{proof}
\end{lm}

\begin{lm}\label{lm:groupwelldefined}
    Let $L/K$ be a field extension with $L\subseteq K_{s}$.
    Then $C:=\wp(A_{L})\cap A_{K}$ is a subgroup of $A_{K}$ with the property that $\wp(A_{K})\subseteq C$.
    \begin{proof}
	Since $A_{K}\subseteq A_{L}$, it follows that $\wp(A_{K})\subseteq \wp(A_{L})$.
	It remains to show that $\wp(A_{K})\subseteq A_{K}$.
	Let $a\in A_{K}$ and let $\sigma \in G$.
	Then using $G$-equivariance of $\wp$ and the fact that $a\in A_{K}$ we have
	\[ \sigma(\wp(a))=\wp(\sigma(a))=\wp(a), \]
	hence $\wp(a)\in A_{K}$ as well.
    \end{proof}
\end{lm}

\begin{lm}\label{lm:coefficients}
    Let $L:=K(\wp^{-1}(C))$ and $L_{i}:=K(\wp^{-1}(C_{i}))$ for each $i\in I$.
    Then
    \[ A_{L}=\bigcup_{i\in I}A_{L_{i}}. \]
    \begin{proof}
	We show first the inclusion $\supseteq $.
	Let $a\in A_{L_{i}}$ for some $i\in I$, i.e.~for all $\sigma\in G(A/\wp^{-1}(C_{i}))$ we have $\sigma(a)=a$.
	Let then $\sigma\in G(A/\wp^{-1}(C))$, which is by definition the set of automorphisms $\tau\in G$ such that $\tau(b)=b$ for all $b\in \wp^{-1}(C)$.
	Since $a\in \wp^{-1}(C_{i})\subseteq \wp^{-1}(C)$, we deduce that $\sigma(a)=a$.

	We move on to the inclusion $\subseteq $.
	Let $a\in A_{L}$, i.e.~$\sigma(a)=a$ for all $\sigma\in \Gal(K_{s}/L)=G(A/\wp^{-1}(C))$.
	Consider the open subgroup $G(A/a)\subseteq G$.
	The corresponding field $K(a)\subseteq K_{s}$ is finite over $K$, as we saw in \Cref{lm:intermediate}.
	And since $\Gal(K_{s}/L)\subseteq G(A/a)$, we have $K(a)\subseteq L$.
	Since $K(a)$ is finite over $K$ we may find some index $i\in I$ such that $K(a)\subseteq L_{i}$, hence
	\[ a\in A_{K(a)}\subseteq A_{L_{i}}. \]
    \end{proof}
\end{lm}

\begin{thm}
    Viewing abelian extensions of $K$ as subfields of $K_{s}$, there is an inclusion-preserving bijection:
    \begin{align*}
	\{ C\trianglelefteq A_{K} \text{ s.t. } \wp(A_{K})\subseteq C \} & \leftrightarrow \{ L/K \text{ abelian w/ exponent dividing } n\}.
    \end{align*}
    Given $C\trianglelefteq A_{K}$ as above, the corresponding field extension is
    \[ \Phi(C):=K(\wp^{-1}(C)); \]
    and conversely, given $L/K$ as above, the corresponding subgroup is
    \[ \Psi(L):=\wp(A_{L})\cap A_{K}. \]
    \begin{proof}
	The functions $\Phi$ and $\Psi$ are well-defined by \Cref{lm:fieldwelldefined} and \Cref{lm:groupwelldefined} respectively.

	Let us check first that $\Psi\circ \Phi=\id$.
	Let $C\subseteq A_{K}$ be a subgroup such that $\wp(A_{K})\subseteq C$ and let $L=K(\wp^{-1}(C))$.
	We denote $\wp(A_{L})\cap A_{K}$ by $C'$, so that the goal is showing that $C'=C$.

	We start with the inclusion $C\subseteq C'$.
	The subgroup $A_{L}\subseteq A$ consists of all $a\in A$ which are fixed by the automorphisms in $\Gal(K_{s}/L)=G(A/\wp^{-1}(C))$, which in turn are precisely the automorphisms fixing all the elements in $\wp^{-1}(C)$.
	Hence $\wp^{-1}(C)\subseteq A_{L}$ and therefore $C\subseteq \wp(A_{L})\cap A_{K}=C'$.

	Moving on to the other inclusion $C'\subseteq C$, the first thing we claim is that $G(A/A_{L})=G(A/\wp^{-1}(C))$.
	Indeed, from $\wp^{-1}(C)\subseteq A_{L}$ we immediately deduce $G(A/A_{L})\subseteq G(A/\wp^{-1}(C))$.
	And conversely, if $\sigma\in G$ fixes every element in $\wp^{-1}(C)$, then it is an automorphism in $\Gal(K_{s}/L)$ by definition.
	But every element $b\in A_{L}$ satisfies $\tau(b)=b$ for all $\tau\in \Gal(K_{s}/L)$, hence $\sigma$ fixes every element in $A_{L}$ as well.
	Hence $G(A/A_{L})=G(A/\wp^{-1}(C))$.
	In particular, $K(A_{L})=K(\wp^{-1}(C))=L$.
	But we have seen already that $C\subseteq C'$, so $\wp^{-1}(C)\subseteq \wp^{-1}(C')$ and
	\[ L=K(\wp^{-1}(C))\subseteq K(\wp^{-1}(C'))\subseteq K(A_{L})=L. \]
	This implies that $K(\wp^{-1}(C))=K(\wp^{-1}(C'))$.
	In particular, if we are in the situation of finite index, then we can apply \Cref{prop:finitecase} to conclude that $C=C'$.
	Indeed, both field extensions have the same degrees, so both $C$ and $C'$ have the same index over $\wp(A_{K})$.
	Equality follows then from the already proven inclusion $C\subseteq C'$.
	This finishes the case in which $C$ has finite index over $\wp(A_{K})$.
	The general case follows from this case thanks to \Cref{lm:coefficients}, because
	\[ C'=\wp(A_{L})\cap A_{K}=\bigcup_{i\in I}\wp(A_{L_{i}})\cap A_{K}=\bigcup_{i\in I}C_{i}'=\bigcup_{i\in I}C_{i}=C. \]
    \end{proof}
\end{thm}

\newpage
\bibliographystyle{alpha}
\bibliography{main}
\vfill

\end{document}
