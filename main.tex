\documentclass[12pt]{amsart}

\input{preamble}

\DeclareMathOperator{\Hom}{Hom}
\DeclareMathOperator{\Gal}{Gal}
\DeclareMathOperator{\Aut}{Aut}
\newcommand{\ot}{\otimes}
\newcommand{\op}{\oplus}
\newcommand{\id}{\mathrm{id}}

\begin{document}

\title{Short notes on Generalized Kummer Theory}
\maketitle

\section{Preliminaries}

\subsection*{Goal:} given a field $K$ and a non-zero natural number $n$, characterize all Galois extensions of $K$ whose Galois group is abelian with exponent $d\mid n$.

\subsection*{Language:} by \textit{abelian} extension we mean a Galois extension $L/K$ with abelian Galois group; by \textit{cyclic} extension we mean a Galois extension $L/K$ with cyclic Galois group.

\subsection*{Reference:} \cite[\S 4.10]{bos18}.

\section{Setting}

\begin{enumerate}
    \item Let $K$ be a field and fix a separable closure $K_{s}$.
    \item Let $n\in \N$ be a non-zero natural number.
    \item Let $G:=\Gal(K_{s}/K)$ be the absolute Galois group.
    \item Let $A$ be an abelian group endowed with the discrete topology and a continuous action of $G$ on $A$ via group automorphisms, which we will denote by $\sigma \cdot a=:\sigma(a)$.
    \item For each intermediate field $K\subseteq L\subseteq K_{s}$ we denote
	\[ A_{L}:=\{ a\in A\mid \sigma(a)=a \text{ for all }\sigma\in \Gal(K_{s}/L)\}. \]
    \item Let $\wp\colon A\to A$ be a $G$-equivariant surjective homomorphism whose kernel, denoted $\mu_{n}$, is a cyclic subgroup of order $n$ of $A_{K}$.
\end{enumerate}

Continuity of the action of $G$ on $A$ ensures that for all $a\in A$ we have
\begin{center}
    \begin{tikzcd}
	G(A/a):=\{ \sigma \in G\mid \sigma(a)=a\}\arrow[open]{r} & G.
    \end{tikzcd}
\end{center}
Hence $G(A/a)$ is also closed in $G$ and corresponds to an intermediate field $K\subseteq K_{s}^{G(A/a)}\subseteq K_{s}$ \cite[4.2/3]{bos18}, let's denote it $K(a)$.

\begin{lm}
    The intermediate field $K(a)$ is a finite extension of $K$.
    \begin{proof}
	Let $\{ L_{i} \}_{i\in I}$ be the direct system of all subfields of $K_{s}$ which are finite field extensions of $K$.
	For each $i\in I$, let us denote by
	\[ f_{i}\colon G\to \Gal(L_{i}/K) \]
	the restriction morphism.
	The topology in $G$ is the coarsest one making all the $f_{i}$ continuous.
	Since each $\Gal(L_{i}/K)$ is a finite group, endowed with the discrete topology, it follows that the topology on $G$ should be the smallest topology in which all fibres of the morphisms $f_{i}$ are open.
	But the fibres of all the $f_{i}$ already form a basis for some topology on $G$, so the topology on $G$ can be explicitly described in terms of this basis.

	Since $G(A/a)$ is open and $\id_{K_{s}}\in G(A/a)$, there is some $i\in I$ such that
	\[ f_{i}^{-1}(f_{i}(\id_{K_{s}}))=\Gal(K_{s}/L_{i})\subseteq G(A/a). \]
	From Galois correspondence we deduce now that
	\[ K\subseteq K(a)\subseteq L_{i}, \]
	hence $K(a)$ is also finite over $K$.
    \end{proof}
\end{lm}

More generally, given a subset $\Delta\subseteq A$ we may consider the subgroup
\[ G(A/\Delta):=\{\sigma\in G\mid \sigma(a)=a \text{ for all }a\in \Delta\}=\cap_{a\in \Delta}G(A/a), \]
which is then a closed subgroup but not necessarily an open subgroup.
In any case we obtain an intermediate field $K\subseteq K_{s}^{G(A/\Delta)}\subseteq K_{s}$, which we will denote by $K(\Delta)$.

If $L/K$ is Galois, then the action of $G$ on $A$ restricts to an action of $G$ on $A_{L}$.
Indeed, let $\tau\in G$, $\sigma\in \Gal(K_{s}/L)$ and $a\in A_{L}$.
Since $\Gal(K_{s}/L)\trianglelefteq G$, there is some $\sigma'\in \Gal(K_{s}/L)$ such that
\[ \sigma\tau(a)=\tau\sigma'(a)=\tau(a), \]
hence $\tau(a)\in A_{L}$.
And by definition $\Gal(K_{s}/L)$ acts trivially on $A_{L}$, so we get an induced action of $G/\Gal(K_{s}/L)$ on $A_{L}$.
Using again that $L/K$ is Galois, we may identify this quotient group with $\Gal(L/K)$, obtaining an action of $\Gal(L/K)$ on $A_{L}$.
We can then talk about the cohomology group $H^{1}(\Gal(L/K),A_{L})$.
A function $f\colon \Gal(L/K)\to A_{L}$ is called a \textit{crossed homomorphism} if for all $\sigma,\tau\in \Gal(L/K)$ we have
\[ f(\sigma\tau)=f(\sigma)+\sigma (f(\tau)). \]
A function $f\colon \Gal(L/K)\to A_{L}$ is called a \textit{principal crossed homomorphism} if there exists some $a\in A_{L}$ such that for all $\sigma \in \Gal(L/K)$ we have
\[ f(\sigma)=\sigma(a)-a. \]
Principal crossed homomorphisms form a subgroup of the group of crossed homomorphisms, and the quotient group is then our first cohomology group $H^{1}(\Gal(L/K),A_{L})$.

We are ready now to state the main assumption on which we will rely:

\begin{axiom}
    For every cyclic extension $L/K$ whose degree divides $n$ we have
    \[ H^{1}(\Gal(L/K),A_{L})=0. \]
\end{axiom}

\section{The pairing associated to a subgroup}

Let $C\subseteq A_{K}$ be a subgroup and consider $\wp^{-1}(C)\subseteq A$.
By $G$-equivariance of $\wp$ and our assumption that $C\subseteq A_{K}$, any $\sigma\in G$ restricts to a homomorphism $\sigma\colon \wp^{-1}(C)\to \wp^{-1}(C)$.
If $\sigma(a)=0$ for $a\in \wp^{-1}(C)$, then
\[ \wp(\sigma(a))=\sigma(\wp(a))=\wp(a)=0, \]
because $\wp(a)\in C\subseteq A_{K}$.
Therefore $a\in \mu_{n}\subseteq A_{K}$, and this implies in turn that $\sigma(a)=a=0$.
So the restriction of $\sigma$ is an injective homomorphism $\wp^{-1}(C)\to \wp^{-1}(C)$.
For $a\in \wp^{-1}(C)$ we have
\[ \sigma(a)-a\in \mu_{n} \]
again by $G$-equivariance of $\wp$ and our assumption that $C\subseteq A_{K}$.
So if $\sigma(a)\in \wp^{-1}(C)$, then
\[ \wp(\sigma(a))=\wp(a)\in C \]
and $a\in \wp^{-1}(C)$ as well, showing that the restriction of $\sigma$ is also surjective.
Hence $\sigma$ restricts to a bijection $\wp^{-1}(C)\to \wp^{-1}(C)$.
We obtain in this manner a group homomorphism
\[ G\to \Aut(\wp^{-1}(C)). \]
The kernel of this group homomorphism is $G(A/\wp^{-1}(C))$ by definition.
It is therefore a normal subgroup of $G$, which means in turn that $K(\wp^{-1}(C))/K$ is a Galois extension with Galois group $G_{C}\cong G/G(A/\wp^{-1}(C))$.

We define now a pairing
\begin{align*}
    G_{C}\times C &\longrightarrow \mu_{n} \\
    (\sigma,c) &\longmapsto \sigma(a)-a, \text{ for }a\in \wp^{-1}(c).
\end{align*}
To check that it is well-defined, pick some other $a'\in \wp^{-1}(c)$.
This elemnt will differ from the previous $a$ by some $b\in \mu_{n}$, hence
\[ \sigma(a')-a'=\sigma(a)+\sigma(b)-a-b=\sigma(a)-a. \]
All good then.
Assume from now on that $\wp(A_{K})\subseteq C$.
We factor then the previous pairing into the pairing that we are interested in:
\begin{align*}
    \langle \cdot,\cdot\rangle \colon G_{C}\times C/\wp(A_{K}) &\longrightarrow \mu_{n} \\
    (\sigma,\bar{c}) & \longmapsto \sigma(a)-a, \text{ for }a\in \wp^{-1}(c).
\end{align*}

\begin{lm}\label{lm:injectivity}
    The pairing $\langle \cdot,\cdot\rangle$ is non-degenerate.
    \begin{proof}
	We have to show that the induced morphisms
	\[ \varphi_{1}\colon G_{C}\to \Hom(C/\wp(A_{K}),\mu_{n}) \quad\text{and}\quad \varphi_{2}\colon C/\wp(A_{K})\to \Hom(G_{C},\mu_{n}) \]
	are injective.

	Suppose that $\sigma\in G_{C}$ is such that $\langle \sigma,\bar{c}\rangle=0$ for all $\bar{c}\in C/\wp(A_{K})$.
	In particular, if $\sigma'\in G$ is a preimage of $\sigma$, then $\sigma(a)=a$ for all $a\in \wp^{-1}(C)$.
	This means precisely that $\sigma'\in G(A/\wp^{-1}(C))$, hence $\sigma=1_{G_{C}}$.

	Suppose now that $c\in C$ is such that $\langle \sigma, \bar{c}\rangle=0$ for all $\sigma\in G_{C}$.
	We want to show that $c\in \wp(A_{K})$, so let $a\in \wp^{-1}(c)$.
	For all $\sigma'\in G$ we have $\sigma'(a)=a$, which means that $a\in A_{K}$ and therefore $\bar{c}=0$.
    \end{proof}
\end{lm}

\begin{lm}\label{lm:finiteness}
    $K(\wp^{-1}(C))/K$ is finite if and only if $(C:\wp(A_{K}))$ is finite.
    \begin{proof}
	Suppose first that $[K(\wp^{-1}(C)):K]$ is finite.
	Then its Galois group $G_{C}$ would be finite as well, so $\Hom(G_{C},\mu_{n})$ is finite.
	But $\varphi_{2}$ is injective by \Cref{lm:injectivity}, so $C/\wp(A_{K})$ must be finite as well.

	Conversely, suppose that $C/\wp(A_{K})$ is finite.
	Again, this implies that $\Hom(C/\wp(A_{K}),\mu_{n})$ is finite, so injectivity of $\varphi_{1}$ shows that $[K(\wp^{-1}(C)):K]$ is finite as well.
    \end{proof}
\end{lm}

\newpage
\bibliographystyle{alpha}
\bibliography{main}
\vfill

\end{document}
